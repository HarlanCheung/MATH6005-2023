\documentclass[12pt, a4paper, oneside]{ctexart}
\usepackage{amsmath, amsthm, amssymb, bm, color, framed, graphicx, hyperref, mathrsfs}
\usepackage{amsfonts}
\usepackage{fancyhdr}
\pagestyle{fancy}
\lfoot{}%这条语句可以让页码出现在下方


\title{\textbf{第一次课程作业}}
\author{张浩然 023082910001}
\date{\today}
\linespread{1.5}
\definecolor{shadecolor}{RGB}{241, 241, 255}
\newcounter{problemname}
\newenvironment{problem}{\begin{shaded}\stepcounter{problemname}\par\noindent\textbf{题目\arabic{problemname}. }}{\end{shaded}\par}
\newenvironment{solution}{\par\noindent\textbf{解答. }}{\par}
\newenvironment{note}{\par\noindent\textbf{题目\arabic{problemname}的注记. }}{\par}

\begin{document}

\maketitle

\begin{problem}
    1、计算

    (1)
    $ 
    \begin{pmatrix}
        cosx & sinx \\
        -sinx & cosx
    \end{pmatrix}^n
    $
    (2)
    $
    \begin{pmatrix}
        1 & 1 \\
        -1 & 1
    \end{pmatrix}^n
    $
    (3)
    $
    \begin{pmatrix}
        a & 1 &   &   &   \\
          & a & 1 &   &   \\
          &   & a & 1 &   \\
          &   &   & a & 1 \\
          &   &   &   & a
    \end{pmatrix}^n
    $
\end{problem}

\begin{solution}
(1).记$A=
\begin{pmatrix}
    cosx & sinx \\
    -sinx & cosx
\end{pmatrix}$,
那么$A \cdot A=
\begin{pmatrix}
    cos2x & sin2x \\
    -sin2x & cos2x
\end{pmatrix}$
由三角函数公式可知,
$A^n=
\begin{pmatrix}
    cosnx & sinnx \\
    -sinnx & cosnx
\end{pmatrix}$
实际上$A$是一个旋转矩阵,$A^n$相当于将旋转角度加倍。

(2).由(1)得

$
\begin{pmatrix}
    1 & 1 \\
    -1 & 1
\end{pmatrix}^n
=
\begin{pmatrix}
    \sqrt{2}cos\frac{\pi}{4} & \sqrt{2}sin\frac{\pi}{4} \\
    -\sqrt{2}sin\frac{\pi}{4} & \sqrt{2}cos\frac{\pi}{4}
\end{pmatrix}^n
=(\sqrt{2})^n
\begin{pmatrix}
    cos\frac{n\pi}{4} & sin\frac{n\pi}{4} \\
    -sin\frac{n\pi}{4} & cos\frac{n\pi}{4}
\end{pmatrix}
$

(3).记$A=
\begin{pmatrix}
    a & 1 &   &   &   \\
      & a & 1 &   &   \\
      &   & a & 1 &   \\
      &   &   & a & 1 \\
      &   &   &   & a
\end{pmatrix}$,$B=
\begin{pmatrix}
     0 & 1 & 0 &  0 & 0  \\
     0 & 0  & 1 & 0  & 0  \\
     0 & 0  & 0 & 1 &  0 \\
     0 & 0  & 0  & 0  & 1 \\
     0 & 0  & 0  & 0  & 0
\end{pmatrix}$
$B^2=
\begin{pmatrix}
    0 & 0 & 1 &  0 & 0  \\
    0 & 0  & 0 & 1  & 0  \\
    0 & 0  & 0 & 0 &  1 \\
    0 & 0  & 0  & 0  & 0 \\
    0 & 0  & 0  & 0  & 0
\end{pmatrix}
B^3=
\begin{pmatrix}
    0 & 0 & 0 &  1 & 0  \\
    0 & 0  & 0 & 0  & 1  \\
    0 & 0  & 0 & 0 &  0 \\
    0 & 0  & 0  & 0  & 0 \\
    0 & 0  & 0  & 0  & 0
\end{pmatrix}
B^4=
\begin{pmatrix}
    0 & 0 & 0 &  0 & 1  \\
    0 & 0  & 0 & 0  & 0  \\
    0 & 0  & 0 & 0 &  0 \\
    0 & 0  & 0  & 0  & 0 \\
    0 & 0  & 0  & 0  & 0
\end{pmatrix}$, 所以$B^n=0(n\geq 5)$.
那么$A=aI+B$,
所以$A^n=(aI+B)^n=(aI)^n+C^1_n(aI)^{(n-1)}B+C^2_n(aI)^{(n-2)}B^2+C^3_n(aI)^{(n-3)}B^3+C^4_n(aI)^{(n-4)}B^4=
\begin{pmatrix}
    a^n &   &   &   &   \\
      & a^n &  &   &   \\
      &   & a^n &  &   \\
      &   &   & a^n &  \\
      &   &   &   & a^n
\end{pmatrix}+
\begin{pmatrix}
     & na^{n-1} &   &   &   \\
      &  & na^{n-1} &   &   \\
      &   &  & na^{n-1} &   \\
      &   &   &  & na^{n-1} \\
      &   &   &   & 
\end{pmatrix}+
\begin{pmatrix}
    &  &  \frac{n(n-1)}{2}a^{n-2} &   &   \\
     &  &  &  \frac{n(n-1)}{2}a^{n-2} &   \\
     &   &  & & \frac{n(n-1)}{2}a^{n-2}  \\
     &   &   &  &  \\
     &   &   &   & 
\end{pmatrix}+
\begin{pmatrix}
    &  &   &  \frac{n(n-1)(n-2)}{6}a^{n-3} &   \\
     &  &  &  & \frac{n(n-1)(n-2)}{6}a^{n-3}  \\
     &   &  & &   \\
     &   &   &  &  \\
     &   &   &   & 
\end{pmatrix}+
\begin{pmatrix}
      &   &   &  \frac{n(n-1)(n-2)(n-3)}{24}a^{n-4} \\
      &  &  &   \\
       &  & &   \\
       &   &  &  \\
       &   &   & 
\end{pmatrix}=
\begin{pmatrix}
    a^n &  na^{n-1} & \frac{n(n-1)}{2}a^{n-2}  &  \frac{n(n-1)(n-2)}{6}a^{n-3} &  \frac{n(n-1)(n-2)(n-3)}{24}a^{n-4} \\
      & a^n & na^{n-1} & \frac{n(n-1)}{2}a^{n-2}  &  \frac{n(n-1)(n-2)}{6}a^{n-3} \\
      &   & a^n & na^{n-1} & \frac{n(n-1)}{2}a^{n-2}  \\
      &   &   & a^n & na^{n-1} \\
      &   &   &   & a^n
\end{pmatrix}
$
\end{solution}

\begin{problem}
    12.设$n$阶矩阵$A$可逆,$x$,$y$是$n$维列向量.如果$(A+xy^*)^{-1}$可逆,证明\textbf{Sherman-Morrison公式}:
    \begin{equation*}
        (A+xy^*)^{-1} = A^{-1}-\frac{A^{-1}xy^*A^{-1}}{1+y^*A^{-1}x}
    \end{equation*}
\end{problem}

\begin{solution}
    假设$B=A^{-1}-\frac{A^{-1}xy^*A^{-1}}{1+y^*A^{-1}x}$
    由于可逆矩阵的唯一性,所以要想证明\textbf{Sherman-Morrison Fomula},即证明$(A+xy^*)B=I$即可.

    $(A+xy^*)(A^{-1}-\frac{A^{-1}xy^*A^{-1}}{1+y^*A^{-1}x})=I-\frac{xy^*A^{-1}}{1+y^*A^{-1}x}+xy^*A^{-1}-\frac{xy^*A^{-1}xy^*A^{-1}}{1+y^*A^{-1}x}$

    由于$y^*A^{-1}x$是一个常量,因此上式可化简为

    $I+xy^*A^{-1}-\frac{xy^*A^{-1}(1+y^*A^{-1}x)}{1+y^*A^{-1}x}=I+xy^*A^{-1}-xy^*A^{-1}=I$
    
\end{solution}

\begin{problem}
    14.(1)设矩阵$A$,$C$均可逆,求分块矩阵$
    \begin{pmatrix}
        A & B \\
        0 & C
    \end{pmatrix}
    $的逆矩阵.
    (2)设矩阵$A$可逆,$D-CA^{-1}B$也可逆,证明分块矩阵$
    \begin{pmatrix}
        A & B \\
        C & D
    \end{pmatrix}
    $ 可逆并求其逆.
\end{problem}

\begin{solution}
    (1)设分块矩阵$
    \begin{pmatrix}
        A & B \\
        0 & C
    \end{pmatrix}
    $的逆矩阵为$
    \begin{pmatrix}
        A_1 & B_1 \\
        D_1 & C_1
    \end{pmatrix}
    $,则$
    \begin{pmatrix}
        A & B \\
        0 & C
    \end{pmatrix}
    \begin{pmatrix}
        A_1 & B_1 \\
        D_1 & C_1
    \end{pmatrix}=
    \begin{pmatrix}
        I & 0 \\
        0 & I
    \end{pmatrix}
    $,即
    $
    \begin{cases}
        AA_1+BD_1=I \\
        AB_1+BC_1=0 \\
        CD_1=0 \\
        CC_1=I
    \end{cases}
    $,解得
    $
    \begin{cases}
        A_1=A^{-1} \\
        B_1= -A^{-1}BC^{-1} \\
        C_1=C^{-1} \\
        D_1=0
    \end{cases}
    $.

    所以分块矩阵$
    \begin{pmatrix}
        A & B \\
        0 & C
    \end{pmatrix}
    $的逆矩阵为$
    \begin{pmatrix}
        A^{-1} & -A^{-1}BC^{-1} \\
        0 & C^{-1} 
    \end{pmatrix}
    $.

    (2)我们根据矩阵求逆的方法,构造
    $
    \begin{pmatrix}
        A & B & \vdots & I & 0 \\
        C & D & \vdots & 0 & I
    \end{pmatrix}
    $
    若经过初等变换使左边变成单位矩阵,那么右边即是逆矩阵。
    首先第一行左乘$A^{-1}$,得
    $
    \begin{pmatrix}
        I & A^{-1}B & \vdots & A^{-1} & 0 \\
        C & D & \vdots & 0 & I
    \end{pmatrix}
    $
    然后第二行减去第一行左乘$C$,得
    $
    \begin{pmatrix}
        I & A^{-1}B & \vdots & A^{-1} & 0 \\
        0 & D-CA^{-1}B & \vdots & -CA^{-1} & I
    \end{pmatrix}
    $
    令$E=D-CA^{-1}B$,第二行左乘$E^{-1}$得,
    $
    \begin{pmatrix}
        I & A^{-1}B & \vdots & A^{-1} & 0 \\
        0 & I & \vdots & -E^{-1}CA^{-1} & E^{-1}
    \end{pmatrix}
    $
    消去$A^{-1}B$得
    $
    \begin{pmatrix}
        I & 0 & \vdots & A^{-1}+A^{-1}BE^{-1}CA^{-1} & -A^{-1}BE^{-1} \\
        0 & I & \vdots & -E^{-1}CA^{-1} & E^{-1}
    \end{pmatrix}
    $
    所求逆矩阵为
    $
    \begin{pmatrix}
        A^{-1}+A^{-1}BE^{-1}CA^{-1} & -A^{-1}BE^{-1} \\
         -E^{-1}CA^{-1} & E^{-1}
    \end{pmatrix}
    $,其中$E=D-CA^{-1}B$.
\end{solution}

\begin{problem}
    17.求下列各矩阵的满秩分解:\\
    (1)$
    A=
    \begin{pmatrix}
    1 & 2 & 3 & 0 \\
    0 & 2 & 1 & -1 \\
    1 & 0 & 2 & 1    
    \end{pmatrix}
    $;
    (2)$
    A=    
    \begin{pmatrix}
        1 & -1 & 1 & 1 \\
        -1 & 1 & -1 & -1 \\
        -1 & -1 & 1 & 1 \\ 
        1 & 1 & -1 & -1   
    \end{pmatrix}
    $.

\end{problem}

\begin{solution}
    (1).$A$经过初等行变换后得

    $A_1=
    \begin{pmatrix}
        1 & 0 & 2 & 1 \\
        0 & 1 & \frac{1}{2} & -\frac{1}{2} \\
        0 & 0 & 0 & 0    
    \end{pmatrix}
    $

    $A$的极大线性无关组构成的矩阵$A_2=    
    \begin{pmatrix}
        1 & 2 \\
        0 & 2  \\
        1 & 0    
    \end{pmatrix}
    $
    所以$A$的满秩分解为

    $A=
    \begin{pmatrix}
        1 & 2 \\
        0 & 2  \\
        1 & 0    
    \end{pmatrix}
    \begin{pmatrix}
        1 & 0 & 2 & 1 \\
        0 & 1 & \frac{1}{2} & -\frac{1}{2} \\ 
    \end{pmatrix}
    $

    (2).
    $A$经过初等行变换后得

    $A_1=
    \begin{pmatrix}
        1 & 0 & 0 & 0 \\
        0 & 1 & -1 & -1 \\
        0 & 0 & 0 & 0 \\ 
        0 & 0 & 0 & 0   
    \end{pmatrix}
    $

    $A$的极大线性无关组构成的矩阵$A_2=    
    \begin{pmatrix}
        1 & -1 \\
        -1 & 1  \\
        -1 & -1  \\
        1 & 1  
    \end{pmatrix}
    $
    所以$A$的满秩分解为

    $A=
    \begin{pmatrix}
        1 & -1 \\
        -1 & 1  \\
        -1 & -1  \\
        1 & 1  
    \end{pmatrix}
    \begin{pmatrix}
        1 & 0 & 0 & 0 \\
        0 & 1 & -1 & -1 \\
    \end{pmatrix}
    $

\end{solution}
% \begin{note}
%     这里是注记. 
% \end{note}

\end{document}