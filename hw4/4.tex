\documentclass[12pt, a4paper, oneside]{ctexart}
\usepackage{amsmath, amsthm, amssymb, bm, color, framed, graphicx, hyperref, mathrsfs}
\usepackage{amsfonts}
\usepackage{fancyhdr}
\pagestyle{fancy}
\lfoot{}%这条语句可以让页码出现在下方


\title{\textbf{第四次课程作业}}
\author{张浩然 023082910001}
\date{\today}
\linespread{1.5}
\definecolor{shadecolor}{RGB}{241, 241, 255}
\newcounter{problemname}
\newenvironment{problem}{\begin{shaded}\stepcounter{problemname}\par\noindent\textbf{题目\arabic{problemname}. }}{\end{shaded}\par}
\newenvironment{solution}{\par\noindent\textbf{解答. }}{\par}
\newenvironment{note}{\par\noindent\textbf{题目\arabic{problemname}的注记. }}{\par}

\begin{document}

\maketitle

\begin{problem}
    5.设
    \begin{equation*}
        A=
        \begin{pmatrix}
            1 & 1 & 2 \\
            0 & 1 & 1 \\
            1 & 3 & 4 
        \end{pmatrix}
    \end{equation*}
    求 A 的四个相关子空间.
\end{problem}


\begin{solution}
    $
    (A,I)=
    \begin{pmatrix}
        1 & 1 & 2 & 1 & 0 & 0 \\
        0 & 1 & 1 & 0 & 1 & 0 \\
        1 & 3 & 4 & 0 & 0 & 1 
    \end{pmatrix}
    \stackrel{elementary transformation}{\longrightarrow}
    \begin{pmatrix}
        1 & 0 & 1 & 1 & -1 & 0 \\
        0 & 1 & 1 & 0 & 1 & 0 \\
        0 & 0 & 0 & -1 & -2 & 1 
    \end{pmatrix}
    =(H_A,P)
    $

    由此,$PA=H_A,r(A)=2$

    可知$A$前两列线性无关

    因此

    $R(A)=span[(1,0,1)^T,(1,1,3)^T]$
    
    $R(A^T)=R(H_A^T)=span[(1,0,1)^T,(0,1,1)^T]$

    $N(A)=N(H_A)=span[(1,1,-1)^T]$

    $N(A^T)=\left\{x|xA=0\right\} = span[(1,2,-1)^T]$


\end{solution}

%%%%%%%%%%%%%%%%%%%%%%%%%%%%%%%%%%%%%%
\begin{problem}
    7.设$V$是所有$n$阶实数矩阵按矩阵的加法和数乘作成的实线性空间,$U$是$V$中所有迹为零的矩阵的集合.证明$U$是$V$的子空间,并求$U$的维数和一个补空间.
\end{problem}


\begin{solution}
    \begin{proof}
        $\forall A,B \in U , tr(A+B)=tr(A)+tr(B)=0$
    
        $\therefore A+B \in U$
    
        $\forall C \in U , \lambda \in \mathbb{F}, tr(\lambda C)=\lambda tr(C)=0$
    
        $\therefore \lambda C \in U$

        $\therefore U$是$V$的子空间.
    \end{proof}
    $dim(V)=n^2$

    由于$trace=0$是线性限制,因此$U$的自由度减小$1$
    
    $dim(U)=n^2-1$

    考虑$W=\left\{cI|c \in \mathbb{R}\right\} $

    显然$W$是$V$的子空间,且$U$和$W$正交

    $dim(W)=1$

    $dim(V)=dim(U)+dim(W)$

    因此$W$即为所求子空间.
\end{solution}
%%%%%%%%%%%%%%%%%%%%%%%%%%%%%%%%%%%%%%%
\begin{problem}
    8.设$V$是所有次数小于$n$的实系数多项式组成的实线性空间,$U=\left\{f(x)\in V:f(1)=0 \right\}$.证明$U$是$V$的子空间,并求$V$的一个补空间.
\end{problem}
\begin{solution}
    \begin{proof}
        $\forall f_1,f_2 \in U, f_1(1)+f_2(1)=0$

        $\therefore f_1+f_2 \in U$

        $\forall f_3 \in U , \lambda \in \mathbb{F}, \lambda f_3(1)=0$

        $\therefore \lambda f_3 \in U$

        因此$U$是$V$的一个子空间.

    \end{proof}
    设 $W:\left\{cx^n|c \in \mathbb{R}\right\} $
    
    易知$W$是$V$的一个子空间

    $dim(V)=n,dim(U)=n-1,dim(W)=1$

    $\therefore dim(V)=dim(U)+dim(W)$

    因此$W$即为所求子空间.
\end{solution}
%%%%%%%%%%%%%%%%%%%%%%%%%%%%%%%%%%%%%%
\begin{problem}
    9.设$U=\left[(1,2,3,6)^T,(4,-1,3,6)^T,(5,1,6,12)^T\right] $,$W=\left[(1,-1,1,1)^T,(2,-1,4,5)^T\right] $是$\mathbb{R}^4$的两个子空间.
    \begin{enumerate}
        \item  求$U\cap W$的基;
        \item 扩充$U\cap W$的基,使其成为$U$的基;
        \item 扩充$U\cap W$的基,使其成为$W$的基;
        \item 求$U+W$的基.
    \end{enumerate}
\end{problem}

\begin{solution}
    (1).令$V=U+W$
    
    $A=
    \begin{pmatrix}
        1 & 4 & 5 & 1 & 2 \\
        2 & -1 & 1 & -1 & -1 \\
        3 & 3 & 6 & 1 & 4 \\
        6 & 6 & 12 & 1 & 5
    \end{pmatrix}$

    经初等行变换后得

    $A \rightarrow
    \begin{pmatrix}
        1 & 0 & 1 & 0 & \frac{7}{9} \\
        0 & 1 & 1 & 0 & -\frac{4}{9} \\
        0 & 0 & 0 & 1 & 3 \\
        0 & 0 & 0 & 0 & 0
    \end{pmatrix}$

    由此得出$(1,2,3,6)^T,(4,-1,3,6)^T,(1,-1,1,1)^T$是 $V$ 的一组基.

    $dim(V)=3$

    易知$dim(U)=2,dim(W)=2$

    因此$dim(U \cap W)=dim(U)+dim(W)-dim(V)=1$

    因此要求$U \cap W$的基,只需要求其一个非零向量即可;

    设$U=[\alpha_1,\alpha_2,\alpha_3],W=[\beta_1,\beta_2]$

    因此$\beta_2$可以由$\alpha_1,\alpha_2,\beta_1$线性表示.

    $x_1\alpha_1+x_2\alpha_2+x_3\beta_1=\beta_2$

    方程组解为$(\frac{7}{9},-\frac{4}{9},3)$

    因此$-3\beta_1+\beta_2=(-1,2,1,2)^T$是$U\cap W$的基.

    (2).$(-1,2,1,2)^T,(1,2,3,6)^T$

    (3).$(-1,2,1,2)^T,(1,-1,1,1)^T$

    (4).$(-1,2,1,2)^T,(1,2,3,6)^T,(1,-1,1,1)^T$
\end{solution}

%%%%%%%%%%%%%%%%%%%%%%%%%%%%%%%%%%%%%%
\begin{problem}
    10.设$U=\left\{(x,y,z,w):x+y+z+w=0\right\},W=\left\{(x,y,z,w):x-y+z-w=0\right\}  $求$U \cap W,U+W$的维数与基.
\end{problem}

\begin{solution}
    $dim(U)=dim(W)=3$

    设$\alpha \in U \cap W$

    $
    \begin{cases}
        x+y+z+w=0\\
        x-y+z-w=0
    \end{cases}
    $

    $\therefore U\cap W$ 的一组基为$(1,0,-1,0),(0,1,0,-1)$

    $\therefore dim(U\cap W)=2$

    $\therefore dim(U+W)=4$

    $\therefore U+W$的一组基为$\mathbb{F}^4$的一组标准基.
\end{solution}
% \begin{note}
%     这里是注记. 
% \end{note}

\end{document}