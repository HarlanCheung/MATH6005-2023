\documentclass[12pt, a4paper, oneside]{ctexart}
\usepackage{amsmath, amsthm, amssymb, bm, color, framed, graphicx, hyperref, mathrsfs}
\usepackage{amsfonts}
\usepackage{fancyhdr}
\pagestyle{fancy}
\lfoot{}%这条语句可以让页码出现在下方


\title{\textbf{第八次课程作业}}
\author{张浩然 023082910001}
\date{\today}
\linespread{1.5}
\definecolor{shadecolor}{RGB}{241, 241, 255}
\newcounter{problemname}
\newenvironment{problem}{\begin{shaded}\stepcounter{problemname}\par\noindent\textbf{题目\arabic{problemname}. }}{\end{shaded}\par}
\newenvironment{solution}{\par\noindent\textbf{解答. }}{\par}
\newenvironment{note}{\par\noindent\textbf{题目\arabic{problemname}的注记. }}{\par}

\begin{document}

\maketitle

\begin{problem}
    6. 设 $ a $ 是复常数, $ V=\left\{\mathrm{e}^{a x} f(x): f(x) \in \mathbb{C}_{n}[x]\right\} $ 是 $ n$ 维复线性空间,
    
    (1) 证明求导运算 $ \partial: \alpha \mapsto \frac{\mathrm{d} \alpha}{\mathrm{d} x} $ 是 $ V $ 上的线性变换;
    
    (2) 求 $ \partial $ 的 $Jordan$ 标准形.
\end{problem}


\begin{solution}
    (1) 设  $v=e^{a x} f(x)$  和  $w=e^{a x} g(x) $ 为  $V $ 中的任意两个元素, 
    
    其中 $ f(x), g(x) \in C^{n}[x]$, $c $ 为任意复数。
    
    首先计算  $\partial(v)=\partial\left(e^{a x} f(x)\right)  $:
    $\partial(v)=\frac{d}{d x}\left(e^{a x} f(x)\right)=e^{a x} \frac{d f(x)}{d x}+a e^{a x} f(x)$
    
    类似地, 对  $w$  应用  $\partial  :\partial(w)=\frac{d}{d x}\left(e^{a x} g(x)\right)=e^{a x} \frac{d g(x)}{d x}+a e^{a x} g(x)$

    现在计算 $ \partial $ 在 $ c v+w$  上的作用:

    $\begin{array}{l}
        \partial(c v+w)=\partial\left(c e^{a x} f(x)+e^{a x} g(x)\right) \\
        =\partial\left(e^{a x}(c f(x)+g(x))\right) \\
        =e^{a x} \frac{d(c f(x)+g(x))}{d x}+a e^{a x}(c f(x)+g(x)) \\
        =c e^{a x} \frac{d f(x)}{d x}+e^{a x} \frac{d g(x)}{d x}+a c e^{a x} f(x)+a e^{a x} g(x) \\
        =c\left(e^{a x} \frac{d f(x)}{d x}+a e^{a x} f(x)\right)+\left(e^{a x} \frac{d g(x)}{d x}+a e^{a x} g(x)\right) \\
        =c \partial(v)+\partial(w)
    \end{array}$
    
    由此证明  $\partial $ 满足线性变换的定义条件。

    (2)$V$的一组基为$\left\{e^{a x}, x e^{a x}, \cdots, x^{n-1} e^{a x}\right\}$

    $\partial$在这组基下的矩阵为:
    
    $
    \left(\begin{array}{ccccc}
        a & 1 & 0 & \cdots & 0 \\
        0 & a & 2 & \cdots & 0 \\
        \vdots & \vdots & \vdots & \ddots & \vdots \\
        0 & 0 & 0 & \cdots & n-1\\
        0 & 0 & 0 & \cdots & a
        \end{array}\right)
    $
    
    由于$A$的特征多项式为$\left(\lambda-a\right)^{n}$,因此$A$的特征值为$a$,且其代数重数为$n$,几何重数为$1$,因此$A$的$Jordan$标准型为:
    
    $    
    \left(\begin{array}{ccccc}
        a & 1 & 0 & \cdots & 0 \\
        0 & a & 1 & \cdots & 0 \\
        \vdots & \vdots & \vdots & \ddots & \vdots \\
        0 & 0 & 0 & \cdots & 1\\
        0 & 0 & 0 & \cdots & a
        \end{array}\right)
    $
    
\end{solution}

%%%%%%%%%%%%%%%%%%%%%%%%%%%%%%%%%%%%%%
\begin{problem}
    18. (本题是幂零矩阵的 $Jordan $标准形定理的证明中, 当数 $ a=\alpha_{1}^{\mathrm{T}} e_{1} \neq 0  $时的实际计算.) 
    
    设$A=\left(\begin{array}{lllll}0 & 2 & 0 & 1 & 0 \\0 & 0 & 1 & 0 & 0 \\0 & 0 & 0 & 0 & 0 \\0 & 0 & 0 & 0 & 1 \\0 & 0 & 0 & 0 & 0\end{array}\right)$,
    
    计算 $ A $ 的 $Jordan$ 标准形.
\end{problem}


\begin{solution}
    
    $
        A^3=0 ,r(A)=3$

    因此$A$的$Jordan$标准型最大的$Jordan$块为 3 阶,有 2 块$Jordan$块.

    根据幂零矩阵的$Jordan$标准型定理,$A$的$Jordan$标准型为
    $J_3\oplus J_2$
\end{solution}
%%%%%%%%%%%%%%%%%%%%%%%%%%%%%%%%%%%%%%%
\begin{problem}
    27. 求下列矩阵的 $Jordan $标准形:
    
    (1) $ \left(\begin{array}{ccc}0 & 1 & 0 \\ -4 & 4 & 0 \\ -2 & 1 & 2\end{array}\right) $;
    (2) $ \left(\begin{array}{ccc}2 & 6 & -15 \\ 1 & 1 & -5 \\ 1 & 2 & -6\end{array}\right)$ ;
    (3) $ \left(\begin{array}{ccc}9 & -6 & -2 \\ 18 & -12 & -3 \\ 18 & -9 & -6\end{array}\right)$ 
\end{problem}
\begin{solution}
    (1) $A$的特征多项式为$\left|\lambda E-A\right|=\left|\begin{array}{ccc}\lambda & -1 & 0 \\ 4 & \lambda-4 & 0 \\ 2 & -1 & \lambda-2\end{array}\right|=\lambda^{3}-6 \lambda^{2}+12\lambda-8=(\lambda-2)^3$

    $A-2I=
    \begin{pmatrix}
        -2 & 1 & 0 \\
        -4 & 2 & 0 \\
        -2 & 1 & 0
    \end{pmatrix}$

    $r(A-2I)=1=r(J-2I)$

    $\therefore J=
    \begin{pmatrix}
        2 & 1 & 0 \\
        0 & 2 & 0 \\
        0 & 0 & 2
    \end{pmatrix}$

    (2)计算方法同(1),$J=\begin{pmatrix}
        -1 & 1 & 0 \\
        0 & -1 & 0 \\
        0 & 0 & -1
    \end{pmatrix}$

    (3)计算方法同(1),$J=\begin{pmatrix}
        -3 & 1 & 0 \\
        0 & -3 & 0 \\
        0 & 0 & -3
    \end{pmatrix}$

\end{solution}
%%%%%%%%%%%%%%%%%%%%%%%%%%%%%%%%%%%%%%
\begin{problem}
    28. 求下列矩阵的 $Jordan$ 标准形, 并求变换矩阵 $ P $, 使 $ P^{-1} A P=J $ :
    
    (1)  $\left(\begin{array}{ccc}-4 & 2 & 10 \\ -4 & 3 & 7 \\ -3 & 1 & 7\end{array}\right)$ ;
    (2)  $\left(\begin{array}{ccc}3 & 2 & 1 \\ 0 & 4 & 0 \\ -1 & 2 & 5\end{array}\right) $;
    (3)  $\left(\begin{array}{cccc}2 & 2 & 2 & 1 \\ -1 & -1 & -3 & -2 \\ 1 & 2 & 5 & 3 \\ -1 & -2 & -4 & -2\end{array}\right)$ .
\end{problem}

\begin{solution}
   (1) $A$的特征多项式为$\left|\lambda E-A\right|=\left|\begin{array}{ccc}\lambda+4 & -2 & -10 \\ 4 & \lambda-3 & -7 \\ 3 & -1 & \lambda-7\end{array}\right|=\lambda^{3}-6 \lambda^{2}+12\lambda-8=(\lambda-2)^3$

    $A-2I=\begin{pmatrix}
        -6 & 2 & 10 \\
        -4 & 1 & 7 \\
        -3 & 1 & 5
    \end{pmatrix}$

    $r(A-2I)=2=r(J-2I)$

    $\therefore J=\begin{pmatrix}
        2 & 1 & 0 \\
        0 & 2 & 1 \\
        0 & 0 & 2
    \end{pmatrix}$

    $P^{-1}AP=J,P^{-1}(A-2I)P=(J-2I)$

    $\therefore
    P=\begin{pmatrix}
        2 & 0 & 1 \\
        1 & 1 & -3 \\
        1 & 0 & 0
    \end{pmatrix}$

    (2)计算方法同(1),
    $J=\begin{pmatrix}
        4 & 1 & 0 \\
        0 & 4 & 0 \\
        0 & 0 & 4
    \end{pmatrix},
    P=\begin{pmatrix}
        -1 & 1 & 2 \\
        0 & 0 & 1 \\
        -1 & 0 & 0
    \end{pmatrix}    
    $

    (3)计算方法同(1),
    $J=\begin{pmatrix}
        2 & 1 & 0 & 0 \\
        0 & 2 & 0 & 0 \\
        0 & 0 & 2 & 1 \\
        0 & 0 & 0 & 2
    \end{pmatrix},
    P=\begin{pmatrix}
        1 & 1 & 1 & 0 \\
        -1 & 0 & -2 & 0 \\
        1 & 0 & 3 & 0 \\
        -1 & 0 & -3 & 1
    \end{pmatrix}
    $


\end{solution}

% \begin{note}
%     这里是注记. 
% \end{note}

\end{document}