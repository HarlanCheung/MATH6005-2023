\documentclass[12pt, a4paper, oneside]{ctexart}
\usepackage{amsmath, amsthm, amssymb, bm, color, framed, graphicx, hyperref, mathrsfs}
\usepackage{amsfonts}
\usepackage{fancyhdr}
\pagestyle{fancy}
\lfoot{}%这条语句可以让页码出现在下方


\title{\textbf{第六次课程作业}}
\author{张浩然 023082910001}
\date{\today}
\linespread{1.5}
\definecolor{shadecolor}{RGB}{241, 241, 255}
\newcounter{problemname}
\newenvironment{problem}{\begin{shaded}\stepcounter{problemname}\par\noindent\textbf{题目\arabic{problemname}. }}{\end{shaded}\par}
\newenvironment{solution}{\par\noindent\textbf{解答. }}{\par}
\newenvironment{note}{\par\noindent\textbf{题目\arabic{problemname}的注记. }}{\par}

\begin{document}

\maketitle

\begin{problem}
    33.在欧式空间$\mathbb{R}^n$中求一个超平面$W$,使得向量$e_1+e_2$在$W$中的最佳近似向量为$e_2$.
\end{problem}


\begin{solution}
    
\end{solution}

%%%%%%%%%%%%%%%%%%%%%%%%%%%%%%%%%%%%%%
\begin{problem}
    37.设$\alpha_0$是欧式空间$V$中的单位向量,$\sigma(\alpha)=\alpha-2(\alpha,\alpha_0)\alpha_0,\alpha \in V$.证明
    
    (1). $\sigma$是线性变换;
    
    (2). $\sigma$是正交变换.
\end{problem}


\begin{solution}
    
\end{solution}
%%%%%%%%%%%%%%%%%%%%%%%%%%%%%%%%%%%%%%%
\begin{problem}
    38.证明:欧氏空间V的线性变换 $ \sigma $ 是反对称变换(即$(\sigma(\alpha),\beta)=-(\alpha,\sigma(\beta))$) $ \Leftrightarrow $ $ \sigma $ 在V
的标准正交基下的矩阵是反对称矩阵.

\end{problem}
\begin{solution}
    
\end{solution}
%%%%%%%%%%%%%%%%%%%%%%%%%%%%%%%%%%%%%%
\begin{problem}
    39.设 $\sigma$ 是实平面 $\mathbb{R}^{2}$上的线性变换, 其关于标准基的矩阵为
    
    P=$\begin{pmatrix}
    c & s \\
    s & -c
    \end{pmatrix}$,
    
    其中 $ c^ {2} $ + $ s^ {2} $ =1.证明 $ \sigma $ 是反射变换,并计算其对称轴.

\end{problem}

\begin{solution}
   
\end{solution}

% \begin{note}
%     这里是注记. 
% \end{note}

\end{document}