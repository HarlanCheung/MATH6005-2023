\documentclass[12pt, a4paper, oneside]{ctexart}
\usepackage{amsmath, amsthm, amssymb, bm, color, framed, graphicx, hyperref, mathrsfs}
\usepackage{amsfonts}
\usepackage{fancyhdr}
\pagestyle{fancy}
\lfoot{}%这条语句可以让页码出现在下方


\title{\textbf{第六次课程作业}}
\author{张浩然 023082910001}
\date{\today}
\linespread{1.5}
\definecolor{shadecolor}{RGB}{241, 241, 255}
\newcounter{problemname}
\newenvironment{problem}{\begin{shaded}\stepcounter{problemname}\par\noindent\textbf{题目\arabic{problemname}. }}{\end{shaded}\par}
\newenvironment{solution}{\par\noindent\textbf{解答. }}{\par}
\newenvironment{note}{\par\noindent\textbf{题目\arabic{problemname}的注记. }}{\par}

\begin{document}

\maketitle

\begin{problem}
    33.在欧式空间$\mathbb{R}^n$中求一个超平面$W$,使得向量$e_1+e_2$在$W$中的最佳近似向量为$e_2$.
\end{problem}


\begin{solution}
    $n=2$时,$W=e_2$

    $n>2$时,要求$e_1$与$W$正交,因此$W$是一个正交于$e_1$并且包含$e_2$的超平面. 
\end{solution}

%%%%%%%%%%%%%%%%%%%%%%%%%%%%%%%%%%%%%%
\begin{problem}
    37.设$\alpha_0$是欧式空间$V$中的单位向量,$\sigma(\alpha)=\alpha-2(\alpha,\alpha_0)\alpha_0,\alpha \in V$.证明
    
    (1). $\sigma$是线性变换;
    
    (2). $\sigma$是正交变换.
\end{problem}


\begin{solution}
(1).证明 $\sigma$ 是线性变换:
 要证明 $\sigma$ 是线性变换, 我们需要证明两个性质: 加法性 $\sigma ( \alpha + \beta) = \sigma (  \alpha )+ \sigma( \beta ) $和齐次性
 $\sigma (c \alpha )=c\sigma( \alpha )$,对所有 $ \alpha , \beta \in V$和所有标量$c$成立.
 
 加法性证明:
 
 $\sigma(\alpha+\beta)=(\alpha+\beta)-2((\alpha+\beta), \alpha _ {0})\alpha _ {0} $
 
 展开并重新排列: $\sigma(\alpha+\beta)= \alpha  -2(  \alpha , \alpha _ {0}  ) \alpha _ {0} + \beta -2(  \beta ,\alpha _ {0}  )  \alpha _ {0} $
 
 这可以重写为: $\sigma ( \alpha + \beta) = \sigma (  \alpha )+ \sigma( \beta ) $
 
 齐次性证明:
 
 $\sigma(c\alpha)=c\alpha-2(c\alpha,\alpha _ {0})\alpha _ {0}$ 
 
 因 $\alpha _ {0}$ 是单位向量,提取$c$: $\sigma(c\alpha)=c(\alpha-2(\alpha,\alpha _ {0})\alpha _ {0})$
 
 这可以重写为: $\sigma(c\alpha) =c\sigma(\alpha)$

 (2).证明 $\sigma$ 是正交变换:

 正交变换是指保持向量内积不变的变换, 即($\sigma$($\alpha$)$\sigma$($\beta$)) =($\alpha$,$\beta$),对所有 $ \alpha $,$\beta$$\in $ $V$.
 
 我们需要证明 $\sigma$ 保持任意两个向量的内积不变。
 
 由于 $ \alpha _ {0} $ 是单位向量,$(\alpha _ {0},\alpha _ {0})=1$.
 
 考虑$(\sigma(\alpha),\sigma(\beta))$ :
 
 $(\sigma(\alpha),\sigma(\beta))=((\alpha-2(\alpha,\alpha _ {0})\alpha _ {0})(\beta-2(\beta,\alpha _ {0})\alpha _ {0}))$
 
 展开内积:$(\sigma(\alpha),\sigma(\beta))=(\alpha,\beta)-2(\alpha,\alpha _ {0})(\alpha _ {0},\beta)-2(\beta,\alpha _ {0})(\alpha,\alpha _ {0})+4(\alpha,\alpha _ {0})(\beta,\alpha _ {0})$
 
 根据 $ \alpha _ {0} $ 的单位性质和内积的分配律, 我们可以简化为:($\sigma$($\alpha$)$\sigma$($\beta$)) =($\alpha$,$\beta$)
 
 从而证明了 $\sigma$ 保持向量内积不变, 所以它是一个正交变换。

\end{solution}
%%%%%%%%%%%%%%%%%%%%%%%%%%%%%%%%%%%%%%%
\begin{problem}
    38.证明:欧氏空间V的线性变换 $\sigma$ 是反对称变换(即$(\sigma(\alpha),\beta)=-(\alpha,\sigma(\beta))$) $ \Leftrightarrow $ $\sigma$ 在V
的标准正交基下的矩阵是反对称矩阵.

\end{problem}
\begin{solution}
    证明:设 $\sigma$ 在$V$的标准正交基 $\alpha _ {1},\alpha _ {2},\cdots,\alpha _ {n}$ 下的矩阵是$A=(a_ {ij})$.则 $\sigma(\alpha _ {i})=\sum _ {k=1}^ {n} a_ {ki} \alpha _ {k} ,1\leqslant i \leqslant  n$.
    
    于是$(\sigma(\alpha _ {i}),\alpha _ {j})= \sum _ {k=1}^ {n}a_ {ki}(\alpha _ {k},\alpha _ {j})= a_ {ji} ,( \alpha _ {i},\sigma (\alpha _ {j}))= \sum _ {k=1}^ {n}a_ {kj}(\alpha _ {i},\alpha _ {k})= a_ {ij} $ .
    因此 $ \sigma $ 是反对称变换当且仅当 $ a_ {ji} $ =- $ a_ {ij} $ ,即A是反对称矩阵。

\end{solution}
%%%%%%%%%%%%%%%%%%%%%%%%%%%%%%%%%%%%%%
\begin{problem}
    39.设 $\sigma$ 是实平面 $\mathbb{R}^{2}$上的线性变换, 其关于标准基的矩阵为
    
    P=$\begin{pmatrix}
    c & s \\
    s & -c
    \end{pmatrix}$,
    
    其中 $ c^ {2} $ + $ s^ {2} $ =1.证明 $\sigma$ 是反射变换,并计算其对称轴.

\end{problem}

\begin{solution}
    \begin{enumerate}
    \item 证明 $\sigma$ 是一个反射变换:
    \begin{itemize}
    \item 对于变换 $\sigma$ 的任意向量 $\alpha = (x, y)$,变换后的向量是 $\sigma(\alpha) = (cx + sy, sx - cy)$。
    \item 如果我们取 $\alpha$ 为 $(c, s)$,则 $\sigma(\alpha) = (c^2 + s^2, sc - sc) = (1, 0) = \alpha$。
    \item 如果我们取 $\alpha$ 为 $(-s, c)$,则 $\sigma(\alpha) = (-cs - sc, s^2 - c^2) = (0, -1) = -\alpha$。
    \item 这表明 $\sigma$ 实际上是沿着由 $(c, s)$ 定义的直线的反射。
    \end{itemize}
    
    \item 计算对称轴:
    \begin{itemize}
    \item 对称轴是保持不变的直线,所以它必须与保持不变的向量 $(c, s)$ 平行。
    \item 对称轴的方程可以写为 $y = mx + b$,其中斜率 $m$ 是向量 $(-s, c)$ 的 $y$ 分量除以 $x$ 分量,即 $m = \frac{c}{-s}$。
    \item 由于 $(c, s)$ 在变换下保持不变,它也位于对称轴上,因此我们可以用它来找到 $b$。
    \item 我们可以设置点 $(0, s)$ 和 $(c, 0)$ 在对称轴上,从而得到 $b = s$ 和 $b = 0$。综合这两点,我们得到对称轴的方程为 $y = \frac{c}{-s} x + s$。
    \end{itemize}
    \end{enumerate}
    
\end{solution}

% \begin{note}
%     这里是注记. 
% \end{note}

\end{document}