\documentclass[12pt, a4paper, oneside]{ctexart}
\usepackage{amsmath, amsthm, amssymb, bm, color, framed, graphicx, hyperref, mathrsfs}
\usepackage{amsfonts}
\usepackage{fancyhdr}
\pagestyle{fancy}
\lfoot{}%这条语句可以让页码出现在下方


\title{\textbf{第三次课程作业}}
\author{张浩然 023082910001}
\date{\today}
\linespread{1.5}
\definecolor{shadecolor}{RGB}{241, 241, 255}
\newcounter{problemname}
\newenvironment{problem}{\begin{shaded}\stepcounter{problemname}\par\noindent\textbf{题目\arabic{problemname}. }}{\end{shaded}\par}
\newenvironment{solution}{\par\noindent\textbf{解答. }}{\par}
\newenvironment{note}{\par\noindent\textbf{题目\arabic{problemname}的注记. }}{\par}

\begin{document}

\maketitle

\begin{problem}
     32.(1).设欧式空间$\mathbb{R}_{[x]_2}$中的内积为
     \begin{equation*}
        (f,g)= \int_{-1}^1 f(x)g(x)\mathrm{d}x
     \end{equation*}
    求基$1,t,t^2$的度量矩阵。
\end{problem}


\begin{solution}
    首先我们计算基之间的内积,代入求得:
    \begin{equation*}
        (1,1)=\int_{-1}^1 1 \mathrm{d}t=2
    \end{equation*}
    \begin{equation*}
        (t,t)=\int_{-1}^1 t^2 \mathrm{d}t= \frac{2}{3}
    \end{equation*}
    \begin{equation*}
        (t^2,t^2)=\int_{-1}^1 t^4 \mathrm{d}t= \frac{2}{5}
    \end{equation*}
    \begin{equation*}
        (1,t)=(t,1)=\int_{-1}^1 t \mathrm{d}t= 0
    \end{equation*}
    \begin{equation*}
        (1,t^2)=(t^2,1)=\int_{-1}^1 t^2 \mathrm{d}t= \frac{2}{3}
    \end{equation*}
    \begin{equation*}
        (t,t^2)=(t^2,t)=\int_{-1}^1 t^3 \mathrm{d}t= 0
    \end{equation*}
    所以度量矩阵为:
    $
    \begin{pmatrix}
        2 & 0 & \frac{2}{3} \\
        0 & \frac{2}{3} & 0 \\
        \frac{2}{3} & 0 & \frac{2}{5} 
    \end{pmatrix}
    $.                                                    
\end{solution}


\begin{problem}
    37.在欧式空间$\mathbb{R}^4$中,求三个向量$\alpha_1=(1,0,1,1)^T$,$\alpha_2=(2,1,0,-3)^T$和$\alpha_3=(1,-1,1,-1)^T$所生成的子空间的一个标准正交基。
\end{problem}


\begin{solution}
    $\alpha_1$标准化得:
    \begin{equation*}
        \beta_1=\alpha_1=(1,0,1,1)^T/\sqrt{3}
    \end{equation*}
    做向量$\alpha_2$在向量$\beta_1$方向上的投影,并做差标准化得:
    \begin{equation*}
        \beta_2=\alpha_2-\frac{(\beta_1,\alpha_2)}{(\beta_1,\beta_1)}\beta_1=(7,3,1,-8)^T/\sqrt{123}
    \end{equation*}
    做向量$\alpha_3$在向量$\beta_1,\beta_2$方向上的投影,并做差标准化得:
    \begin{equation*}
        \beta_3=\alpha_3-\frac{(\beta_1,\alpha_3)}{(\beta_1,\beta_1)}\beta_1-\frac{(\beta_2,\alpha_3)}{(\beta_2,\beta_2)}\beta_2=(3,54,-23,20)^T/\sqrt{3854}
    \end{equation*}
    $\beta_1,\beta_2,\beta_3$即为一组标准正交基.
\end{solution}

\begin{problem}
    42.设线性空间$V=\mathbb{R}^2$是欧式空间(未必是通常的欧式空间).设$\alpha_1=(1,1)^T,\alpha_2=(1,-1)^T$与$\beta_1=(0,2)^T,\beta_2=(6,12)^T$是$V$的两组基.设$\alpha_j$与$\beta_k$的内积分别为
    \begin{equation*}
        (\alpha_1,\beta_1)=1,(\alpha_1,\beta_2)=15,(\alpha_2,\beta_1)=-1,(\alpha_2,\beta_2)=3
    \end{equation*}

    (1).求两组基的度量矩阵;

    (2).求$V$的一个标准正交基.
\end{problem}
\begin{solution}
    (1).利用内积的性质可得
    \begin{equation*}
        (\alpha_1,\alpha_1)=(\alpha_1,\frac{1}{6}\beta_2-\frac{1}{2}\beta_1)=2
    \end{equation*}
    同理,$(\alpha_1,\alpha_2)=1,(\alpha_2,\alpha_2)=2,(\beta_1,\beta_1)=2,(\beta_1,\beta_2)=12,(\beta_2,\beta_2)=126$
    所以两组基的度量矩阵分别为:
    $
    \begin{pmatrix}
        2 & 1 \\
        1 & 2
    \end{pmatrix}
    $
    ,
    $
    \begin{pmatrix}
        2 & 12 \\
        12 & 126
    \end{pmatrix}
    $

    (2).
    将$\alpha_1$标准化后得$\gamma_1=(1,1)^T/\sqrt{2}$
    做向量$\alpha_2$在向量$\gamma_1$方向上的投影,并做差得:
    \begin{equation*}
        \gamma_2=\alpha_2-\frac{(\gamma_1,\alpha_2)}{(\gamma_1,\gamma_1)}\gamma_1=(\frac{1}{2},-\frac{3}{2})
    \end{equation*}
    \begin{equation*}
        (\gamma_2,\gamma_2)=(\alpha_2-\frac{1}{2}\alpha_1,\alpha_2-\frac{1}{2}\alpha_1)=(\alpha_2,\alpha_2)-(\alpha_2,\alpha_1)+\frac{1}{4}(\alpha_1,\alpha_1)=\frac{3}{2}
    \end{equation*}
    $(\frac{\sqrt{2}}{2},\frac{\sqrt{2}}{2})^T,(\frac{\sqrt{6}}{6},-\frac{\sqrt{6}}{2})^T$是$V$的一个标准正交基.
\end{solution}

\begin{problem}
    44.设$A$是\textbf{反对称}实矩阵(即$A^T=-A$),证明:

    (1).$A$的特征值为$0$或纯虚数;

    (2).设$\alpha+\beta i$是$A$的属于一个非零特征值的特征向量,其中$\alpha,\beta$均为实向量,则$\alpha$与$\beta$正交.

\end{problem}

\begin{solution}
    \begin{proof}
        (1).设$A$的特征值和特征向量分别为$\lambda, v$
        可知
        \begin{equation*}
            Av=\lambda v
        \end{equation*}
        同时取共轭转置得
        \begin{equation*}
            (Av)^*=v^*A^*=\overline{\lambda}v^* =-v^*A
        \end{equation*}
        所以
        \begin{equation*}
            -v^*Av=\overline{\lambda}v^*v
            -\lambda v^*v=\overline{\lambda}v^*v
        \end{equation*}
        由于$v^*v\neq 0$
        所以$\lambda$是纯虚数或$\lambda=0$

        (2).设$\lambda=ki,(k\neq 0)$
            \begin{equation*}
                ki(\alpha+\beta i)=A(\alpha+\beta i)
                A\alpha+A\beta i= ki\alpha-k\beta
            \end{equation*}
            因此:
            \begin{equation*}
                A\alpha=-k\beta
            \end{equation*}
            两边取转置得:
            \begin{equation*}
                \alpha^TA^T=-k\beta^T
            \end{equation*}
            \begin{equation*}
                -\alpha^TA=-k\beta^T
            \end{equation*}
            \begin{equation*}
                \alpha^TA\alpha=k\beta^T\alpha
            \end{equation*}
            \begin{equation*}
                -k\alpha^T\beta=k\beta^T\alpha
            \end{equation*}
            因此$\alpha^T\beta=\beta^T\alpha=0$,即$\alpha,\beta$正交.
    \end{proof}
\end{solution}

\begin{problem}
    设$A$是$Hermite$矩阵.如果对任意向量$x$均有$x^*Ax=0$,则$A=0$.
\end{problem}

\begin{solution}
    \begin{proof}
        设$A$的特征值为$\lambda_i$,对应的特征向量为$v_i$.所以,
        \begin{equation*}
            v_i^*Av_i=\lambda_i v_i^* v_i=0
        \end{equation*}
        由于$v_i$是特征向量不为零,所以$\lambda_i=0$

        把矩阵$A$进行特征值分解得
        \begin{equation*}
            A=V\Lambda V^*
        \end{equation*}
        由于$\Lambda=0$,所以$A=0$
    \end{proof}
\end{solution}
% \begin{note}
%     这里是注记. 
% \end{note}

\end{document}