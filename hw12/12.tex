\documentclass[12pt, a4paper, oneside]{ctexart}
\usepackage{amsmath, amsthm, amssymb, bm, color, framed, graphicx, hyperref, mathrsfs}
\usepackage{amsfonts}
\usepackage{fancyhdr}
\pagestyle{fancy}
\lfoot{}%这条语句可以让页码出现在下方


\title{\textbf{第十二次课程作业}}
\author{张浩然 023082910001}
\date{\today}
\linespread{1.5}
\definecolor{shadecolor}{RGB}{241, 241, 255}
\newcounter{problemname}
\newenvironment{problem}{\begin{shaded}\stepcounter{problemname}\par\noindent\textbf{题目\arabic{problemname}. }}{\end{shaded}\par}
\newenvironment{solution}{\par\noindent\textbf{解答. }}{\par}
\newenvironment{note}{\par\noindent\textbf{题目\arabic{problemname}的注记. }}{\par}

\begin{document}

\maketitle

\begin{problem}
    32. (1) 设$ A=\left(\begin{array}{cc}
        0 & 0 \\
        1 & -2
        \end{array}\right) $, 求  $\mathrm{e}^{A}, \sin A, \cos A $ ;
        
        (2) 已知  $J=\left(\begin{array}{llll}
            -2 & & & \\
            & 1 & 1 & \\
            & & 1 & \\
            & & & 2
            \end{array}\right) $, 求 $ \mathrm{e}^{J}, \sin J, \cos J$ . 
\end{problem}


\begin{solution}
    
\end{solution}

%%%%%%%%%%%%%%%%%%%%%%%%%%%%%%%%%%%%%%
\begin{problem}
    35. 对下列方阵  $A$ , 求矩阵函数  $\mathrm{e}^{A t} $ :
(1)  $A=\left(\begin{array}{ccc}2 & -2 & 3 \\ 1 & 1 & 1 \\ 1 & 3 & -1\end{array}\right)$ 
(2)  $A=\left(\begin{array}{ccc}0 & 1 & 0 \\ 0 & 0 & 1 \\ -8 & -12 & -6\end{array}\right)$ ,
(3)  $A=\left(\begin{array}{ccc}-2 & 1 & 3 \\ 0 & -3 & 0 \\ 0 & 2 & -2\end{array}\right)$ .
\end{problem}


\begin{solution}
    
\end{solution}
%%%%%%%%%%%%%%%%%%%%%%%%%%%%%%%%%%%%%%%
\begin{problem}
    36. 求下列两类矩阵的矩阵函数:  $\cos A, \sin A, \mathrm{e}^{A}$  :

    (1) $ A $ 为幂等矩阵;

    (2)  $A $ 为对合矩阵 (即  $A^{2}=I $ ).
\end{problem}
\begin{solution}
    
\end{solution}
%%%%%%%%%%%%%%%%%%%%%%%%%%%%%%%%%%%%%%
\begin{problem}
    37. 设函数矩阵 $ A(t)=\left(\begin{array}{ccc}
        \sin t & \cos t & t \\
        \frac{\sin t}{t} & \mathrm{e}^{t} & t^{2} \\
        1 & 0 & t^{3}
        \end{array}\right) $, 其中  $t \neq 0$ . 计算  $\lim _{t \rightarrow 0} A(t), \frac{\mathrm{d}}{\mathrm{d} t} A(t), \frac{\mathrm{d}^{2}}{\mathrm{~d} t^{2}} A(t) $. 
\end{problem}

\begin{solution}
   
\end{solution}

% \begin{note}
%     这里是注记. 
% \end{note}

\end{document}