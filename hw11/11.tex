\documentclass[12pt, a4paper, oneside]{ctexart}
\usepackage{amsmath, amsthm, amssymb, bm, color, framed, graphicx, hyperref, mathrsfs}
\usepackage{amsfonts}
\usepackage{fancyhdr}
\pagestyle{fancy}
\lfoot{}%这条语句可以让页码出现在下方


\title{\textbf{第十一次课程作业}}
\author{张浩然 023082910001}
\date{\today}
\linespread{1.5}
\definecolor{shadecolor}{RGB}{241, 241, 255}
\newcounter{problemname}
\newenvironment{problem}{\begin{shaded}\stepcounter{problemname}\par\noindent\textbf{题目\arabic{problemname}. }}{\end{shaded}\par}
\newenvironment{solution}{\par\noindent\textbf{解答. }}{\par}
\newenvironment{note}{\par\noindent\textbf{题目\arabic{problemname}的注记. }}{\par}

\begin{document}

\maketitle

\begin{problem}
    2. 证明: 若  $x \in \mathbb{C}^{n}$ , 则
    
    (1)  $\|x\|_{2} \leqslant\|x\|_{1} \leqslant \sqrt{n}\|x\|_{2}$ ;
    
    (2)  $\|x\|_{\infty} \leqslant\|x\|_{1} \leqslant n\|x\|_{\infty}$ ;
    
    (3)  $\|x\|_{\infty} \leqslant\|x\|_{2} \leqslant \sqrt{n}\|x\|_{\infty}$ .
\end{problem}


\begin{solution}
    (1). $x \in \mathbb{C}^n$

    $||x||_{2}=\sqrt{\sum |x_i|^2}$

    $||x||_{1}=\sum |x_i|$

    $||x||_{1}^2=(\sum |x_i|)^2 \geqslant \sum|x_i|^2 =||x||_2^2$(当且仅当$x_i=0,i=1,2,3,\cdots,n$时等号成立)

    由均值不等式

    $\frac{\sum\left|x_{i}\right|}{n} \leqslant \sqrt{\frac{\sum\left|x_{i}\right|^{2}}{n}}$

    即:$||x||_1 \leqslant \sqrt{n} \cdot ||x||_2$

    $\therefore ||x||_2 \leqslant ||x||_1 \leqslant \sqrt{n} \cdot ||x||_2$

    (2).


    $\|x\|_{\infty}  =\max \left\{\left|x_{i}\right|\right\}$ 

    记$\left|x_{k}\right|  =\max \left\{\left|x_{i}\right|\right\}$

    则 $\|x\|_{\infty}  =\left|x_{k}\right|$ 

    $\|x\|_{1}  =\left|x_{k}\right|+\sum_{i \neq k}^{n}\left|x_{i}\right| $
    
    $n\|x\|_{\infty}  =n\left|x_{k}\right| \geq \sum_{i=1}^{n}\left|x_{i}\right|$
    
    
    
    可知 $\|x\|_{\infty} \leq\|x\|_{1} \leq n \cdot\|x\|_{\infty}$

   (3). 记$\|x\|_{\infty}  =\left|x_{k}\right|$ 

        则 $\|x\|_{\infty}^{2}  =\left|x_{k}\right|^{2}$ 

        $\|x\|_{2}^{2}  =\sum\left|x_{i}\right|^{2} \geq\left|x_{k}\right|^{2}$ 

        $\left(\sqrt{n}\left\|_{x}\right\|_{\infty}\right)^{2}  =n \cdot\left|x_{k}\right|^{2} \geq \sum_{i=1}^{n}\left|x_{i}\right|^{2} $
    
        $\therefore\|x\|_{\infty}  \leq\|x\|_{2} \leq \sqrt{n}\|x\|_{\infty}$

\end{solution}

%%%%%%%%%%%%%%%%%%%%%%%%%%%%%%%%%%%%%%
\begin{problem}
    11. 验证矩阵的极大列和范数与极大行和范数均满足次乘性.
\end{problem}


\begin{solution}
    极大列和范数的次乘性证明

    对于矩阵  $A \in \mathbb{R}^{m \times n} $ 和  $B \in \mathbb{R}^{n \times p} $, 我们有:

    $\|A B\|_{1}=\max _{1 \leq k \leq p} \sum_{i=1}^{m}\left|\sum_{j=1}^{n} a_{i j} b_{j k}\right|$


    利用三角不等式和绝对值的性质, 可以得到:

    $\sum_{i=1}^{m}\left|\sum_{j=1}^{n} a_{i j} b_{j k}\right| \leq \sum_{i=1}^{m} \sum_{j=1}^{n}\left|a_{i j}\right|\left|b_{j k}\right|$


    接着, 我们可以利用$ Holder $不等式, 得到:

    $\|A B\|_{1} \leq \max _{1 \leq k \leq p} \sum_{j=1}^{n}\left|b_{j k}\right| \max _{1 \leq j \leq n} \sum_{i=1}^{m}\left|a_{i j}\right|$


    由于  $\max _{1 \leq k \leq p} \sum_{j=1}^{n}\left|b_{j k}\right| $ 是  $B $ 的极大列和范数, $ \max _{1 \leq j \leq n} \sum_{i=1}^{m}\left|a_{i j}\right| $ 是  $A $ 的极大列和范数, 因此我们有:

    $\|A B\|_{1} \leq\|A\|_{1}\|B\|_{1}$

    这证明了极大列和范数满足次乘性.

    极大行和范数同理.
\end{solution}
%%%%%%%%%%%%%%%%%%%%%%%%%%%%%%%%%%%%%%%
\begin{problem}
    26. 设 $ A_{k}=\left(\begin{array}{cc}\frac{1}{k^{2}} & \frac{k^{2}+k}{k^{2}+1} \\2 & \left(1-\frac{2}{k}\right)^{k}\end{array}\right) $, 求 $ \lim\limits_{k \to \infty}  A_{k} $.
\end{problem}
\begin{solution}
        $A_{k}=\left(\begin{array}{cc}
        \frac{1}{k^{2}} & \frac{k^{2}+k}{k^{2}+1} \\
        2 & \left(1-\frac{2}{k}\right)^{k}
        \end{array}\right)
        $
        
        $\lim _{k \rightarrow \infty} \frac{1}{k^{2}}=0 $

        $\lim _{k \rightarrow \infty} \frac{k^{2}+k}{k^{2}+1}=1 $

        $\lim _{k \rightarrow \infty}\left(1-\frac{2}{k}\right)^{k}=\left(\lim _{k \rightarrow \infty}\left(1-\frac{2}{k}\right)^{\frac{k}{2}}\right)^{2}=\frac{1}{e^{2}} $

        $\therefore \lim _{k \rightarrow \infty} A_{k}=\left(\begin{array}{cc}
        0 & 1 \\
        2 & \frac{1}{e^{2}}
        \end{array}\right)$
\end{solution}
%%%%%%%%%%%%%%%%%%%%%%%%%%%%%%%%%%%%%%
\begin{problem}
   28. 若 $ \lim\limits_{n \to \infty}  A_{n}=B$,则$B$为幂等矩阵.
\end{problem}

\begin{solution}
    $\lim _{n \rightarrow \infty} A^{n}=B $

    $B^{2}=\left(\lim _{n \rightarrow \infty} A^{n}\right)^{2}=\lim _{n \rightarrow \infty} A^{2 n}=B$
\end{solution}

%%%%%%%%%%%%%%%%%%%%%%%%%%%%%%%%%%%%%%

\begin{problem}
    30. 设 $ A=\left(\begin{array}{cc}2 & -\frac{1}{2} \\2 & 0\end{array}\right)$ , 求 $ \sum_{k=0}^{\infty} \frac{A^{k}}{2^{k}} $. 
\end{problem}

\begin{solution}

        
        $|\lambda I-A|=\left|\begin{array}{cc}
        \lambda-2 & \frac{1}{2} \\
        -2 & \lambda
        \end{array}\right|=\lambda(\lambda-2)+1=(\lambda-1)^{2} $

        $A-I=\left(\begin{array}{cc}
        1 & -\frac{1}{2} \\
        2 & -1
        \end{array}\right) \quad r(A-I)=1 $

        $m_{A}(\lambda)=(\lambda-1)^{2}$
        
        
        
        则  $f $ 在 $A$  的谱上的数值为  $\left\{f(1) , f^{\prime}(1)\right\} $
        
        设  $f(t)=\sum_{k=0}^{\infty} \frac{t^{k}}{2^{k}}, g(t)=a_{0}+a_{1} t   $
        
        $f(A)=g(A) \Leftrightarrow f $. $g $ 在A 的谱上的数值相等
        
        
        $f(1)=2$ 
       
        $ f^{\prime}(1)=\sum_{k=1}^{\infty} \frac{k \cdot 1^{k-1}}{2^{k}}=\sum_{k=1}^{\infty} \frac{k}{2^{k}}=\lim _{n \rightarrow \infty}\left(\left(\sum_{i=0}^{n} \frac{1}{2^{i}}\right)-\frac{n}{2^{n}}\right)=2 $
        
        $\therefore  g(1)=a_{0}+a_{1}=2$
        
        $g^{\prime}(1)=a_{1}=2 $
        
        $\therefore a_{0}=0 $
        
        $\sum_{k=0}^{\infty} \frac{A^{k}}{2^{k}}=f(A)=g(A)=2 A=\left(\begin{array}{cc}
        4 & -1 \\
        4 & 0
        \end{array}\right)$
        
\end{solution}

%%%%%%%%%%%%%%%%%%%%%%%%%%%%%%%%%%%%%%
\begin{problem}
    31.设 $A=\left(\begin{array}{ccc}
        -0.6 & 1 & 0.8 \\
        0 & 0.2 & 0 \\
        -0.6 & 1 & 0.8
        \end{array}\right) $. 试判断 $ A $是否幂收敛. 
\end{problem}

\begin{solution}
    $A=\left(\begin{array}{ccc}
        -0.6 & 1 & 0.8 \\
        0 & 0.2 & 0 \\
        -0.6 & 1 & 0.8
        \end{array}\right) $

        $|\lambda I-A|=\left|\begin{array}{ccc}
        \lambda+0.6 & -1 & -0.8 \\
        0 & \lambda-0.2 & 0 \\
        0.6 & -1 & \lambda-0.8
        \end{array}\right|=(\lambda-0.2)\left|\begin{array}{cc}
        \lambda+0.6 & -0.8 \\
        0.6 & \lambda-0.8
    \end{array}\right|=\lambda(\lambda-0.2)^{2}$

    $|\lambda|<1$
    
    $\therefore$ $J_A$幂收敛$\Leftrightarrow A$幂收敛   。  
        
\end{solution}
% \begin{note}
%     这里是注记. 
% \end{note}

\end{document}