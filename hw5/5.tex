\documentclass[12pt, a4paper, oneside]{ctexart}
\usepackage{amsmath, amsthm, amssymb, bm, color, framed, graphicx, hyperref, mathrsfs}
\usepackage{amsfonts}
\usepackage{fancyhdr}
\pagestyle{fancy}
\lfoot{}%这条语句可以让页码出现在下方


\title{\textbf{第五次课程作业}}
\author{张浩然 023082910001}
\date{\today}
\linespread{1.5}
\definecolor{shadecolor}{RGB}{241, 241, 255}
\newcounter{problemname}
\newenvironment{problem}{\begin{shaded}\stepcounter{problemname}\par\noindent\textbf{题目\arabic{problemname}. }}{\end{shaded}\par}
\newenvironment{solution}{\par\noindent\textbf{解答. }}{\par}
\newenvironment{note}{\par\noindent\textbf{题目\arabic{problemname}的注记. }}{\par}

\begin{document}

\maketitle

\begin{problem}
     15.若$\sigma(\alpha_1),\sigma(\alpha_2),\cdots ,\sigma(\alpha_s)$线性相关,证明或否定$\alpha_1,\alpha_2,\cdots,\alpha_s$也线性相关.
\end{problem}


\begin{solution}
    假设 $\sigma\left(\alpha_1\right), \sigma\left(\alpha_2\right), \ldots, \sigma\left(\alpha_s\right)$ 线性相关。这意味着存在不全为零的系数 $c_1, c_2, \ldots, c_s$ 使得
    $$
    c_1 \sigma\left(\alpha_1\right)+c_2 \sigma\left(\alpha_2\right)+\ldots+c_s \sigma\left(\alpha_s\right)=0
    $$

    设 $\sigma$ 是一个线性变换, 我们可以提取系数:
    $$
    \sigma\left(c_1 \alpha_1+c_2 \alpha_2+\ldots+c_s \alpha_s\right)=0
    $$

    如果 $c_1 \alpha_1+c_2 \alpha_2+\ldots+c_s \alpha_s$ 是非零向量, 那么我们就得到了 $\sigma$ 将一个非零向量映射到零向量, 这意味着 $\sigma$ 不是单射, 即不是一一映射.

    如果 $c_1 \alpha_1+c_2 \alpha_2+\ldots+c_s \alpha_s$ 是零向量,那么这意味着 $\alpha_1, \alpha_2, \ldots, \alpha_s$ 线性相关.

    因此不能确定是否线性相关.
\end{solution}

%%%%%%%%%%%%%%%%%%%%%%%%%%%%%%%%%%%%%%
\begin{problem}
    25.分别求导数运算 $\partial: f(x) \mapsto f^{\prime}(x)$ 在标准基 $1, x, x^2, \cdots, x^{n-1}$ 与基 $1,(x-a),(x-$ $a)^2, \cdots,(x-a)^{n-1}$ 下的矩阵. 问 $\partial$ 的行列式与迹是多少? 解释之.
\end{problem}


\begin{solution}
    对于标准基 $\left\{1, x, x^2, \ldots, x^{n-1}\right\}$, 导数运算 $\partial$ 的作用是:
    $$
    \begin{aligned}
    \partial(1) & =0 \\
    \partial(x) & =1 \\
    \partial\left(x^2\right) & =2 x \\
    \vdots & \\
    \partial\left(x^{n-1}\right) & =(n-1) x^{n-2}
    \end{aligned}
    $$

    将这些结果用标准基表示, 矩阵 $A$ 为:
    $$
    A=\begin{pmatrix}
    0 & 1 & 0 & \cdots & 0 \\
    0 & 0 & 2 & \cdots & 0 \\
    0 & 0 & 0 & \ddots & 0 \\
    \vdots & \vdots & \vdots & \ddots & n-2 \\
    0 & 0 & 0 & \cdots & 0
    \end{pmatrix}
    $$

    对于基 $\left\{1,(x-a),(x-a)^2, \ldots,(x-a)^{n-1}\right\}$ ,导数运算 $\partial$ 的作用是:
    $$
    \begin{aligned}
    \partial(1) & =0 \\
    \partial(x-a) & =1 \\
    \partial\left((x-a)^2\right) & =2(x-a) \\
    \vdots & \\
    \partial\left((x-a)^{n-1}\right) & =(n-1)(x-a)^{n-2}
    \end{aligned}
    $$

    将这些结果用 $\left\{1,(x-a),(x-a)^2, \ldots,(x-a)^{n-1}\right\}$ 基表示,矩阵 $B$ 为:
    $$
    B=\begin{pmatrix}
    0 & 1 & 0 & \cdots & 0 \\
    0 & 0 & 2 & \cdots & 0 \\
    0 & 0 & 0 & \ddots & 0 \\
    \vdots & \vdots & \vdots & \ddots & n-2 \\
    0 & 0 & 0 & \cdots & 0
    \end{pmatrix}
    $$

    行列式和迹均为 $0$,导数运算只是降低了多项式的阶数,并没有缩放或者拉伸这个空间.
\end{solution}
%%%%%%%%%%%%%%%%%%%%%%%%%%%%%%%%%%%%%%%
\begin{problem}
    27.(1)求例2.2.22中的幂零变换$\tau$的幂零指数及其在标准基下的矩阵;
    
    (2)设$\sigma,\tau \in EndV$分别是线性空间$V$的同构变换和幂零变换,证明$\sigma+\tau$是$V$的同构变换;

    (3)设$A,D$是可逆矩阵,$B,C$是幂零矩阵,证明分块矩阵$
    \begin{pmatrix}
        A & B \\
        C & D
    \end{pmatrix}$可逆.
    
    \textbf{例 2.2.22}设 $V=\mathbb{F}^n$. 对 $\alpha=\left(x_1, x_2, \cdots, x_n\right)^{\mathrm{T}}$, 定义
    $$
    \sigma(\alpha)=\left(x_1, 0, \cdots, 0\right)^{\mathrm{T}} ; \quad \tau(\alpha)=\left(x_2, x_3, \cdots, x_n, 0\right)^{\mathrm{T}} .
    $$
    
    则 $\sigma$ 与 $\tau$ 均是 $V$ 的线性变换, 且 $\sigma$ 是幂等变换, $\tau$ 是幕零变换.(幂零指数是多少?)
\end{problem}

\begin{solution}
    (1).$$\alpha=\left(x_1, x_2 \cdots x_n\right)^T$$
    $$
    \begin{aligned}
    & T^{n-1}(\alpha)=\left(x_n, 0 \cdots 0\right)^T\\
    & T^n(\alpha)=0 \\
    & \therefore T \text { 的幂零指数为 } n
    \end{aligned}
    $$
    在标准基下的矩阵为
    $$
    A=\begin{pmatrix}
        0 & 1 & 0 & \cdots & 0 \\
        0 & 0 & 1 & \cdots & 0 \\
        0 & 0 & 0 & \ddots & 0 \\
        \vdots & \vdots & \vdots & \ddots & 1 \\
        0 & 0 & 0 & \cdots & 0
    \end{pmatrix}
    $$
    (2).没证出来
    (3).没证出来
\end{solution}
%%%%%%%%%%%%%%%%%%%%%%%%%%%%%%%%%%%%%%
\begin{problem}
   29.设 $V=\mathbb{R}^3, \sigma(x, y, z)=(x+2 y-z, y+z, x+y-2 z)$. 求

    (1) $\sigma$ 的核与像空间的基与维数;
    
    (2) $\sigma$ 的行列式与迹.
\end{problem}

\begin{solution}
    (1).求核空间的基,即为解方程组
    $$
    \begin{aligned}
        x+2y-z=0 \\
        y+z=0 \\
        x+y-2z=0
    \end{aligned}
    $$
    求得$$x=3z,y=-z$$

    因此核空间的基为$(3,-1,1)$,维数为 $1$.

    $\sigma$用矩阵表示为$
    A=\begin{pmatrix}
        1 & 2 & -1 \\
        0 & 1 & 1 \\
        1 & 1 & -2
    \end{pmatrix}$

    计算$A$的列空间得

    像空间的基为$(1,0,1)^T,(2,1,1)^T$,维数是 2

    (2). $det(A)=0,tr(A)=0$
\end{solution}

%%%%%%%%%%%%%%%%%%%%%%%%%%%%%%%%%%%%%%%%
\begin{problem}
    32. 设 $V=\mathbb{R}[x]_n$, 其上的内积为
$$
(f(x), g(x))=\int_0^1 f(x) g(x) \mathrm{d} x .
$$

设 $U=\{f(x) \in V: f(0)=0\}$.

(1) 证明 $U$ 是 $V$ 的一个 $n-1$ 维子空间, 并求 $U$ 的一组基;

(2) 当 $n=3$ 时, 求 $U$ 的正交补 $U^{\perp}$.
\end{problem}

\begin{solution}
    (1).   
    \begin{enumerate}
        \item \textbf{非空}:\( U \) 包含零多项式,因为零多项式在 \( x = 0 \) 处的值为 0.
        \item \textbf{封闭性}:对于任意 \( f(x), g(x) \in U \) 和任意标量 \( \alpha, \beta \),我们有 \( \alpha f(x) + \beta g(x) \in U \).这是因为 \( (\alpha f + \beta g)(0) = \alpha f(0) + \beta g(0) = 0 \).
        \item \textbf{标量乘法}:对于任意 \( f(x) \in U \) 和任意标量 \( \alpha \),我们有 \( \alpha f(x) \in U \).这是因为 \( (\alpha f)(0) = \alpha f(0) = 0 \).
    \end{enumerate}

    \( U \) 的一个基是 \( \{x, x^2, \ldots, x^{n-1}\} \).

    (2).
    当 \( n = 3 \) 时,\( V \) 是所有次数不超过 2 的多项式的集合,\( U \) 的一组基是 \( \{x, x^2\} \).

    $$
    \begin{aligned}
    \int_{0}^{1} f(x) \cdot x \, dx &= 0 \\
    \int_{0}^{1} f(x) \cdot x^2 \, dx &= 0
    \end{aligned}
    $$

    其中 \( f(x) \) 是一个次数不超过 2 的多项式,可以表示为 \( f(x) = ax^2 + bx + c \).

    解方程组得 \( a = \frac{10}{3}c \) 和 \( b = -4c \),其中 \( c \) 是任意实数.

    因此,\( U^\perp \) 是由多项式 \( \frac{10}{3}x^2 - 4x + 1 \) 张成的一维空间.

\end{solution}
% \begin{note}
%     这里是注记. 
% \end{note}

\end{document}